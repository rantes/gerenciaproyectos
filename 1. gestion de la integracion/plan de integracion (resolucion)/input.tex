\usepackage{fancyhdr}
\usepackage[includeheadfoot,footskip=.5cm]{geometry}
\usepackage[small,bf]{caption}
%\usepackage{mathtools}
\usepackage{multirow}
\usepackage{array}
\usepackage{color, colortbl}
%
\pagestyle{fancy}

\geometry{lmargin=2cm,rmargin=3cm,tmargin=0.5cm,bmargin=4cm}
\def\arraystretch{1.5}

% Coloca una línea en los encabezados
\fancyhf{}
\fancyhead[L]{}
\fancyhead[R]{}
\fancyhead[C]{
	\begin{table}[H]
		\centering
		\begin{tabular}{ c c }
		\multirow{3}{*}{
			\includegraphics[width=0.3\textwidth]{images/logoecci.png}}
		 & ESCUELA COLOMBIANA DE CARRERAS INDUSTRIALES \\
		 &	GERENCIA DE PROYECTOS \\
		 &	\hfill {\small GESPRO01} \\
		\end{tabular}
	\end{table}
}
% para mostrar el paginado
\fancyfoot[R]{\thepage}
\renewcommand{\headrulewidth}{0cm}
% Cambia la estructura de una página en blanco
\fancypagestyle{plain}{
\fancyhead[L]{}
\fancyhead[R]{}
\fancyhead[C]{
	\begin{table}[H]
		\centering
		\begin{tabular}{ c c }
		\multirow{3}{*}{
			\includegraphics[width=0.25\textwidth]{images/logoecci.png}}
		 & ESCUELA COLOMBIANA DE CARRERAS INDUSTRIALES \\
		 & GERENCIA DE PROYECTOS \\
		 &	\hfill {\small GESPRO01} \\
		\end{tabular}
	\end{table}
}
\renewcommand{\headrulewidth}{0pt}
}

% Cambia el ancho del encabezado
%\setlength{\headwidth}{16.5cm}

% Cambia el espacio para el encabezado
\setlength{\voffset}{5pt}
\setlength{\headheight}{70pt}
\setlength{\headsep}{0.5cm}

% Cambia el margen de los pies de figura
\setlength{\captionmargin}{10pt}

% Borra la palabra Capítulo del \chaptermark:
\renewcommand{\chaptermark}[1]{\markboth{\MakeUppercase
{\thechapter. #1}}{}}
% Quita la palabra capitulo
\addto\captionsspanish{\renewcommand{\chaptername}{}}

%definicion de colores
\definecolor{LightGrey}{gray}{0.9}
\definecolor{lgray}{gray}{0.75}
\definecolor{Grey}{gray}{0.7}

% Definiendo tamanos de los titulos
\titleformat{\chapter}
	{\bfseries\normalfont\normalfont}
	{\thechapter.}
	{6pt}
	{}
\titlespacing*{\chapter}{0pt}{0pt}{0pt}

\titleformat{\section}[block]
	{\Large\normalfont}
	{\thesection.}
	{6pt}
	{}
\titlespacing*{\section}{0pt}{0pt}{0pt}

\titleformat{\subsection}
{\normalfont\large\bfseries}{\thesubsection}{12pt}{}
\titleformat{\subsubsection}
{\normalfont\normalsize\bfseries}{\thesubsubsection}{12pt}{}
\titleformat{\paragraph}[runin]
{\normalfont\normalsize\bfseries}{\theparagraph}{12pt}{}
\titleformat{\subparagraph}[runin]
{\normalfont\normalsize\bfseries}{\thesubparagraph}{12pt}{}

%definiendo el modo de insercion de capitulos
\titleclass{\chapter}{top}

%definicion de recuadros gris y texto negro para secciones y capitulos
\titleformat{\chapter}[hang]{\Large\normalfont}{%
}{0em}{%
    {%
        \setlength{\fboxsep}{0pt}%
        \colorbox{lgray}{\makebox[\textwidth]{\Large\strut}}%
    }%
    \hspace*{-\textwidth}%
        {\thechapter}%
        \hspace*{1em}%
}[]
%
%\titleformat{\section}[hang]{\Large\normalfont}{6pt}{%
%    {%
%        \setlength{\fboxsep}{0pt}%
%        \colorbox{lgray}{\makebox[\textwidth]{\Large\strut}}%
%    }%
%    \hspace*{-\textwidth}%
%        {\thesection}%
%        \hspace*{1em}%
%}[]