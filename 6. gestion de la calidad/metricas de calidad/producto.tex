\chapter{M\'ETRICAS DEL PRODUCTO}
%
\section{Disponibilidad de la plataforma}
%
Se define como el tiempo de uptime que permanece disponible la plataforma en funcionamiento para los
usuarios.\\%
%
Este factor de calidad es relevante porque el servicio debe funcionar constantemente 7/24/365.\\
%
Por otro lado, existen eventos de mantenimiento que pueden detener la disponibilidad, pero son
programados en las horas cuyo flujo de datos sea el menor posible (por ejemplo: 2:00 a.m. - 4:00 a.m.).\\[0.5cm]%
%
\textbf{M\'etodo de medici\'on.}
%
\begin{enumerate}
	\item Se establece un servicio de watchdog que permite recibir alertas de cuando la plataforma no est\'a
		en funcionamiento y cuando restablece operaciones.
	\item Se lleva registro de todos los incidentes.
	\item Se calcula el tiempo no operacional.
	\item Se genera informe de tiempos no operacionales a lo largo del mes con sus respectivas razones.
	\item Se revisar\'a el informe con el Sponsor y se tomar\'an las acciones correctivas y/o preventivas
		pertinentes.
	\item Se informar\'a al cliente de dichas acciones de ser el caso.
\end{enumerate}
%
\textbf{Resultados deseados.}
%
\begin{enumerate}
	\item El factor downtime no puede superar 10 horas al mes.
\end{enumerate}