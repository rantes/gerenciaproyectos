\chapter{T\'ECNICAS Y HERRAMIENTAS}
%
\begin{table}[H]
	\centering
	\begin{tabular}{| m{4cm} | p{5cm} | p{6cm} |}
	\hline
	\textbf{PROCESO} & \textbf{T\'ECNICAS / HERRAMIENTAS} & \textbf{DESCRIPCI\'ON} \\ \hline
	5.1. Estimaci\'on de los costos del proyecto & Juicio de expertos & Desarrollan la estimaci\'on de los costos.
		Emiten otros juicios en los costos de acuerdo a la experiencia. \\
	\cline{2-3}
	& Estimaci\'on an\'aloga & Se toma como referencia los costos de otros proyectos de la misma envergadura 
		(tama\~no y complejidad).\\
	\cline{2-3}
	& Estimaci\'on por tres variables & Se toma en cuenta la incertidumbre y el riesgo, de  acuerdo a las variables
		de costo esperado, optimista, probable y pesimista.\\
	\hline
	5.2. Determinaci\'on del presupuesto & Suma de costos & Se suma de las estimaciones por entregables. Suma por 
		niveles superiores seg\'un la WBS. Sumar todo el proyecto.\\ 
	  \cline{2-3}
	 & An\'alisis de reserva & Se realiza reservas por contingencias ya sea por asignaci\'on cambios no planificados 
	 	en el alcance y en los costos del proyecto.\\
	\cline{2-3}
	& Juicios de expertos & Experiencia de las \'areas de aplicaci\'on seg\'un la organizaci\'on y los 
		interesados.\\
	\hline
	5.3. Controlar los costos & Gesti\'on  del valor Ganado (EVM) & permite medir el desempe\~no del  cronograma del
		costo en el proyecto.\\
	\cline{2-3}
	& \'Indice de desempe\~no del trabajo por completar (TCPI) & Permite medir la proyecci\'on calculada del 
		desempe\~no del costo que debe lograrse para el trabajo restante, con el prop\'osito de cumplir con el 	
		proyecto.\\
	\cline{2-3}
	& Revisiones del rendimiento del proyecto & Se pretende comparar el rendimiento del costo a lo largo del tiempo, 
		las actividades del cronograma, los hitos vencidos y los alcanzados.\\
	\hline
	\end{tabular}
	
\end{table}