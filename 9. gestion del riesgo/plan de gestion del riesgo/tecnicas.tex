\chapter{T\'ECNICAS Y HERRAMIENTAS}
%
\begin{table}[H]
	\centering
	\begin{tabular}{| m{5cm} | c | p{5cm} |}
	\hline
	\rowcolor{LightGrey}
	\textbf{PROCESO} & \textbf{T\'ECNICAS / HERRAMIENTAS} & \textbf{DESCRIPCI\'ON} \\ \hline
	5.1. Desarrollo del plan de gesti\'on & Lluvia de ideas & Definici\'on del grupo de actividades y herramientas 
	necesarias para la fase de integraci\'on del proyecto.\\
	\hline
	5.2. Desarrollo del Scope Statement & Alcance del proyecto & Con el Alcance se pueden definir cada uno de los 
	requerimientos y las funcionalidades que se quieren.\\ 
	  \cline{2-3}
	 & Juicio de expertos & Se identifica los Requerimientos con sus caracter\'isticas, prioridad y restricciones 
	 de los entregables. \\
	\hline
	5.3. Desarrollo diagrama WBS & Juicio de expertos & Con el consentimiento  de cada uno de los expertos en el 
	desarrollo del proyecto  se podr\'a finalizar exitosamente.\\
	\cline{2-3}
	& Alcance del proyecto & Tener en cuenta  los  requerimientos que el cliente quiere.\\
	\hline
	5.4. Desarrollo diccionario WBS & Documento de stakeholders & Se tiene en cuenta los interesados en el proyecto 
	y la funci\'on que cumple cada uno.\\
	\cline{2-3}
	& WBS & Tener la estructura de las fases y sus actividades.\\
	\cline{2-3}
	& Project charter & Se define la descripci\'on de cada integrante.\\
	\cline{2-3}
	& Juicio de expertos & Con esta t\'ecnica permitir\'a desarrollar quien intervendr\'a en cada una de las 
	actividades.\\
	\hline
	\end{tabular}
	
\end{table}